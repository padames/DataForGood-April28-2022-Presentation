% catvssubcat.tex


\begin{figure}[htbp!]
    \scriptsize
    \centering
    \begin{tikzpicture}
    \begin{axis}[mapfigs]
        \addplot[color=black, mark=pentagon]
            coordinates {
            (3,25.78)(5,19.93)(7,16.02)(9,13.49)(11,11.97)
            };\label{plot:cat}
        \addplot[color=black, mark=triangle*]
            coordinates {
            (3, 16.953333333333323)
            (5, 13.151865186518658)
            (7, 10.679988586545726)
            (9, 9.030561344812489)
            (11, 7.953089315970777)
            };\label{plot:scat}
        \node [draw,fill=white] at (rel axis cs: 0.732,0.762) {\shortstack[l]{
            \ref{plot:cat} {\scriptsize cat/nc/256} \\
            \ref{plot:scat} {\scriptsize scat/nc/256} }};
    \end{axis}
    \end{tikzpicture}
    \caption{\scriptsize Effect of using categories or subcategories on MAP at different top k values for samples of 10,000 users. Key: cat=categories for feature computation, scat= subcategories instead of categories for feature computation, nc=no feature for country of user, 256=size of the cause embedding vectors.}
\end{figure}
